\documentclass[twocolumn,linenumbers]{aastex631}

% Note to self: Don't forget to start up Zotero when you start writing!
% Consider adding the VScode Zotero extension?

\begin{document}

\title{Bayesian Inference and MCMC Methods in Astrophysics}
\author{Agastya Gaur}
\affiliation{University of Illinois at Urbana-Champaign}

\begin{abstract}
  This is the abstract of the paper. It summarizes the work in a concise form.
\end{abstract}

\tableofcontents

\section{Introduction}
In the 4th century BC, Hipparchus, attempting to estimate the length of a year, found the middle of the range of a scattered set of Babylonian solstice measurements. Though an acheivement in its own right for the time, Hipparchus's measurement marked the beginning of what would become a long standing marriage between astronomy and statistics. In the centuries to come, a number of breakthroughs in astrostatistics would continue to occur, with Brahe successfully using the mean of a dataset to increase precision of measurements and Laplace rediscovering the work of Thomas Bayes and applying his statistical theories extensively to astronomical problems. Most notably, in the early 1800s, Legendre developed least squares parameter estimation to model the orbit of comets \citep{feigelsonStatisticalChallengesModern2003}. By the end of the 19th century, astronomy had firmly established itself as a quantitative science, driven by the refinement of statistical methods to confront the uncertainties inherent in measurement.

The next 100 years brought two developments that reshaped this tradition: the rise of physics as the explanatory foundation of astronomy, and the advent of computing, which enabled unprecedented scales of quantitative analysis. As astronomy grew increasingly intertwined with the theories of physics, the field transformed into what we now call astrophysics. This shift did not replace the statistical tradition but expanded it, integrating new forms of quantitative reasoning with physical modeling. Advances in computing increased the scale of the statistical analysis that was feasible to perform, and since then it has only been rising. While the early history of the field was dominated by statistical reasoning, the growth of physics and computation broadened this into what we now call quantitative analysis (QA): a synthesis of statistical inference, numerical modeling, and data-driven computation.Today, astrophysics sits in a landscape of complex statistical problems that demand new quantitative approaches and more computing power by the day. It is fair to say that QA has become the backbone of research in modern astrophysics.

\bibliography{references}

\end{document}
