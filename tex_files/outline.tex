\documentclass[12pt]{article}

\usepackage{hyperref}

\begin{document}

\textbf{Title:} Bayesian Inference and MCMC Methods in Astrophysics

\begin{enumerate}
  \item Introduction
  \begin{enumerate}
    \item Quantitative analysis is the backbone of Astrophysics.
    \item Focusing on Bayesian and MCMC in this paper bc of their popularity in the field, flexibility, and compatibility with many types of problems.
  \end{enumerate}
  \item Methodology \\ (vonToussant, Brewer)
  \begin{enumerate}
    \item Bayesian inference essentials
    \item MCMC methods \\ For both of these, explain how they work, maybe include the math and the code, as well as practical guidance on how to apply them to simple problems.
  \end{enumerate}
  \item Case Studies
  \begin{enumerate}
    \item Bayesian Frameworks for Exoplanet detection \\ (Ruffio)
    \begin{enumerate}
      \item Problem
      \item Methods
      \item Pros/Cons
      \item Extensions
    \end{enumerate}
    \item Cosmological Parameter Estimation with MCMC Methods \\ (Akeret) \\ Notes: Make this even more specialized. Read some more about it
    \item Bayesian approach to Gravity wave detection \\ (Littenberg) \\ Notes: Don't go too deep into this. Only talk about detection, don't get into fitting the graviatational waveform.
  \end{enumerate}
  \item Conclusion
  \begin{enumerate}
    \item State of the field
    \item Areas for future work
    \item 
  \end{enumerate}
\end{enumerate}


\end{document}